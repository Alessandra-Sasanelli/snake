\documentclass{article}
\usepackage{geometry}[margin=2cm]
\usepackage{graphicx}

\title{Snake}
\author{Alessandra Sasanelli, Simone Maccario}
\date{\today}

\begin{document}
	\maketitle
	\abstract{In this document we would like to explain how the game works so that you can play it. In particular we want to focus on the logic and describe the main keyboard input and the initial state of the appstate to run and play it.}
	
	\section{The game}
	Our goal for this project was to recreate the snake game and we believe we succeeded.
	The game consists of 2 basic elements, which are a snake and an apple. The goal of the game? Simple, eat as many apples as possible in order to increase the length of your snake. To complicate the game and not create default pathners, the apple's position is generated by a random function while the speed of the snake will increase based on how many apples are eaten.
	There are two different ways to lose: the first is when the snake hits itself, heii!! The snake must not eat the body, it hurts a lot; the second is when the snake hits one of the four edges... poor snake its head is sore, who knows if it will still want to play.
	To interact with the application we thought of using simple inputs that refer to our same old inputs from when we played on our first mobile phones. Which ones are they? Just a moment, give me time.
	Furthermore, we tried to make the game more fun by using a particular and human sound, when the snake eats, and, to alleviate the defeat, we used a ridiculous game-over.
	
	\section{Command}	
	Before we tell you all our keyboard inputs you need to know how to run the game. First of all you need to run our code in the Dr.Racket application and write this function to run it: \textbf{\emph{(snake-game SNAKE APPLE GAME-F QUIT-F TICK RATE}} where:
	
	\begin{itemize}
		\item Snake represents the default one, which is a snake three units long with initial movement to the right
		\item Apple is generated in a random position in the background
		\item Game is turned off because when the game runs it draws a sort of home page
		\item Quit is deactivated so that the application is not closed
		\item Tick counts every tick of the clock
		\item Rate represents how many speed's level there are
	\end{itemize}
	
	or simply \textbf{\emph{(snake-game DEFAULT)}}, where DEFAULT is the same appstate written before.
	\newpage
	Well, now it's time to get to know our commands:
	
	\begin{itemize}
		\item "s" -$>$ using the letter s you can start the game, then the home page disappears and the game draws the game canvas.
		\item "$\uparrow$" -$>$ using the up arrow you can change the direction of the snake up.
		\item "$\rightarrow$" -$>$ using the right arrow you can change the direction of the snake to the right.
		\item "$\downarrow$" -$>$ using the down arrow you can change the snake's direction to down
		\item "$\leftarrow$" -$>$ using the left arrow you can change the direction of the snake to the left.
		\item "r" -$>$ using the Exit button you can exit the game and the game window closes.page.
		\item "esc" -$>$ using the escape button you can quit the game and the game's window is closed
	\end{itemize}
	
	\large{\textbf{IMPORTANT:}}\emph{If you try to change direction in the opposite direction, the snake will not change direction.}

\end{document}