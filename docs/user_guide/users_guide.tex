\documentclass{article}
\usepackage{geometry}[margin=0cm]
\usepackage{graphicx}

\title{Snake}
\author{Alessandra Sasanelli, Simone Maccario}
\date{\today}

\begin{document}
	\maketitle
	\abstract{The document would like to explain how the game works so a person can easily play. In particular we want to focus on the logic and describe the main keyboard inputs and the initial appstate to run and play it.}
	
	\section{The game}
	
	The goal of the project was to recreate the snake game and we believe we succeeded.
	The game consists into two basic elements, that are the snake and the apple and the game's aim is that the snake eats as many apples as possible to increase his length.
	To make the game more complicate and without having predefined paths, the position of the apple is randomly generated by a function, while the speed of the snake will increase based on how many apples are eaten.
	There are two different ways to lose: the first is when the snake hits itself... heyyy!! the snake must not eat its body, it hurts a lot; the second is when the snake hits one of the four edges... poor snake has a sore head, who knows if it will still want to play.
	To interact with the application, we thought to use simple inputs that are similar to the old inputs used when we played with our first mobile phones. Which ones are they? Just a moment, give me time.
	Furthermore, we tried to make the game more fun by using a particular and human sound when the snake eats, while when the player loose, we used a ridiculous game-over.
	
	\section{Command}	
	
	Before to explain the keyboard inputs, you need to know how to run the game. First of all you need to run our code in the Dr.Racket application and write this function to run it: \textbf{\emph{(snake-game SNAKE APPLE GAME-F QUIT-F TICK RATE)}} where:
	
	\begin{itemize}
		\item Snake represents the default one, which is a snake three units long with initial movement to the right.
		\item Snake represents by default is a snake made and long by three units “block” with initial movement to the right.
		\item Apple appears in a random position in the background.
		\item Game is turned off because when the game runs it draws some sort of home page.
		\item Quit is deactivated so that the application is not closed.
		\item Tick counts every tick of the clock.
		\item Rate represents how many speed's level there are.
	\end{itemize}
	
	\noindent and all of these attributes make a default appstate, so if you want you can just write the default function like this \textbf{\emph{(snake-game DEFAULT)}}, where DEFAULT represents all the attributes written before.

	Well, now it's time to get to know our commands:
	
	\begin{itemize}
		\item "s" -$>$ using the letter s you can start the game, then the home page disappears and the game draws the game canvas.
		\item "$\uparrow$" -$>$ using the up arrow you can change the direction of the snake up.
		\item "$\rightarrow$" -$>$ using the right arrow you can change the direction of the snake to the right.
		\item "$\downarrow$" -$>$ using the down arrow you can change the snake's direction to down
		\item "$\leftarrow$" -$>$ using the left arrow you can change the direction of the snake to the left.
		\item "r" -$>$ using the letter r you can restart the game, then the game canvas disappears and the application redraws its home page.
		\item "esc" -$>$ using the esc button you can exit the game and the game window closes.
	\end{itemize}
	
	\noindent\large{\textbf{IMPORTANT:}}\emph{If you try to change direction in the opposite direction, the snake will not change direction.}

\end{document}